%%%%%%%%%%%%%%%%%%%%%%%%%%%%%%%%%%%%%%%%%%%%%%%%%%%%%%
%% Auteur	: Julien Hébert-Doutreloux
%% Date		: 2020/05/01
%% Mail		:
%% Git		: https://github.com/Jhebertaz/
%%%%%%%%%%%%%%%%%%%%%%%%%%%%%%%%%%%%%%%%%%%%%%%%%%%%%
%% Réference
%% https://en.wikibooks.org/wiki/LaTeX
%% https://fr.wikibooks.org/wiki/LaTeX
%% https://www.overleaf.com/
%%https://www.latex-project.org/help/documentation/amsldoc.pdf
%%http://mirrors.ibiblio.org/CTAN/info/lshort/french/lshort-fr.pdf
%%https://cs.overleaf.com/learn
%%%%%%%%%%%%%%%%%%%%%%%%%%%%%%%%%%%%%%%%%%%%%%%%%%%%%
\documentclass[french, babel,twocolumn]{article}
%% Pour le français
\usepackage [french]{babel}
\usepackage [utf8]{inputenc}
\usepackage [T1] {fontenc}
\usepackage{xspace}
%% Marge et mise en page
\usepackage{layout}
\usepackage{lscape}
%% Mise en page
\usepackage[letterpaper]{geometry}
%% American Mathematical Society Package 
\usepackage{amsmath}
\usepackage{amsfonts}
\usepackage{amssymb}
\usepackage{amsthm}
\usepackage{amstext}
%% Environnement IEEEeqnarray
\usepackage[retainorgcmds]{IEEEtrantools}
%% Divers 
\usepackage{graphicx}
\usepackage{subfig}
\usepackage{float}
\usepackage{caption}
\usepackage{multicol}
\usepackage{array}
\usepackage{lipsum}
\usepackage{hyperref}
%% Opérateur
\DeclareMathOperator{\argh}{argh}
\DeclareMathOperator*{\Nut}{Nut}
%% Modification/Définition
\newcolumntype{L}{>{$}l<{$}}
\newcolumntype{V}{>{\ttfamily\textbackslash}l}
%% Information
\title{La pratique sous \LaTeX}
\author{Julien Hébert-Doutreloux\\%
	Comité de l'Axiomatique%
	\thanks{Département de Mathématique}, Université de Montréal}
\date{\today}
\begin{document}
	\onecolumn
	%% Affiche la page présentation
	\maketitle
	%% Table des matière
	\tableofcontents
	%% Espace vertical
	%\vspace{1cm}
	\section{Introduction}
		Juste pour la forme une succincte introduction du système de composition de documents qu'est \LaTeX. Avec ce derniers, la rédaction de document se fait à l'aveugle par opposition au traitement de texte usuelle fait par des éditeurs tels \textit{Microsoft Word} où l'on voit au fur et à mesure l'évolution du document. À l'aveugle n'insinue pas \textit{a contrario} que l'on travail avec une déficience. En fait, la conception esthétique d'un travail est souvent source de distraction. Même si la forme d'un document importe à celui qui le rédige, elle est secondaire. De prime abord, le contenu d'un text à préséance sur l'aspect visuelle de celui-ci, en cela la rédaction par une série de commande permet à l'usager de se concentrer sur le fond et non la forme. Le prix à payer pour produire avec \LaTeX{} est l'apprentissage d'un code (la courbe d'apprentissage est plutôt abupt pour ceux et celles qui n'ont jamais fait de programmation).
	\twocolumn
	\section{Les bases}
	Une bonne connaissance des bases que sont les symboles réservés, la syntaxe, les environnements ainsi que les diverses commandes intégrées est nécessaires pour être en mesure d'appréhender les différents exemples de cette initiation à \LaTeX.
%	
	\subsection{Les caractères réservés}
	Un trait commun à nombre de langages de programmation est bien évidemment l'existence de caractères spéciaux. Ces caractères ont, pour la plupart, des utilisations spécifiques. Notons ces symboles suivants :
%	
	\begin{center}
		\# \$ \% \^{} \& \_ \{ \} \~{}  \textbackslash
	\end{center}
%
	\par Et voici, comment ils sont respectivement générés suivant les commandes ci-dessous :
	\begin{flushleft}\label{lst:Caratères réservés}
		\verb|\# \$ \% \^{} \& \_ \{ \} \~{}|
		\verb|\textbackslash|
	\end{flushleft}
%	
	\subsection{La structure}
	Un article scientifique est ordonné par une structure suivant une logique de sectionnement des différentes parties du contenu et il en est de même pour un document rédigé sous \LaTeX. Seulement une préparation est requise avant d'entreprendre l'écriture. Il s'agit d'un préambule regroupant minimalement la classe du document (article, book, report, ou une classe personnalisée) ainsi qu'accessoirement certaines extensions (très utile pour la rédaction d'équations mathématiques). Pour la \textbf{classe} \emph{article} de même que pour toute autres classes, la structure est:
%	
	\begin{table}
		\centering
		\begin{tabular}{l}
			\hline
			\verb|\documentclass[|\textit{options}\verb|]|\verb|{article}|\\
			\verb|\usepackage{|\textit{package}\verb|}|\\
			\vdots\\
			\verb|\begin{document}|\\
			\verb|	Hello World|\\
			\vdots\\
			\verb|\end{document}|\\
			\hline
		\end{tabular}\\
	\end{table}
%
	\begin{center}
		\verb|\documentclass[|\textit{options}\verb|]|\verb|{article}|
	\end{center}
%
	\par La première commande du préambule (ci-dessus) est celle
	indiquant le type de document dont les \textit{options} permettent de modifier le comportement et le style du fichier lors de sa sortie. Ces options trouvent toutes leurs utilités lors de la rédaction en une langue différente de l'anglais. Il en est de même pour ce document adapté à la langue française avec pour options \verb|[french, babel]|. Ensuite pour charger les diverses \textbf{extensions}, on utilise la commande \verb|\usepackage{|\textit{package}\verb|}|. Une fois le préambule complété,  le corps de l'article tient respectivement entre les commandes \verb|\begin{document}| puis \verb|\end{document}|. L'entièreté du contenu de l'article s'y trouve et se qui vient après n'est pas pris en compte. Afin de conserver une clarté au fils des paragraphes, \LaTeX{}  propose plusieurs commandes de sectionnement \textit{text} est le titre de la section.
%	
	\begin{table}
		\centering
		\begin{tabular}{l}
			\verb|\section{|\textit{text}\verb|}|\\
			\verb|\subsection{|\textit{text}\verb|}|\\
			\verb|\subsubsection{|\textit{text}\verb|}|\\
		\end{tabular}
	\end{table}
%	
	\par Ces commandes ont aussi la possibilité d'avoir un titre plus court de telle sorte qu'une section ayant un très long titre puisse rentrer dans la table des matières. L'option s'insère entre l'argument et la commande de la même façon que la commande précisant le comportement du document. L'option permet de donner un titre concis pour la table de matières.
%	
	\subsection{Références croisées}
	%			Environnements
	%			Listes, énumérations et descriptions
	%			verbatim
	%			Tableaux/matrices
	%			supertabular ou longtable
	%			figure
	%			caption
	%			Protection des commandes “fragiles”
	\section{Formules mathématiques}
	Les mathématiques est \LaTeX{} l'un des  atouts essentiels de  puisque  la rédaction de \LaTeX formule est somme tout assez simple. L'un des groupes d'extensions pour écrire des équations avancées est \AmS-\LaTeX. Cet ensemble d'extension est produit par \emph{American Mathematical Society}. Il suffit d'inscrire dans le préambule.
%
	\begin{verbatim}
		\usepackage{amsmath}
		\usepackage{amsfonts}
		\usepackage{amssymb}
		\usepackage{amsthm}
	\end{verbatim}
%
	\textbf{Note} : Bien connaître son alphabet grec est un atout à prendre en considération, et ce afin d'écrire efficacement des équations. (cf.~Table~\ref{greekLetters}),
	%% La définition des colonnes aurait pu être changée pour alléger le code
	%% Organiser les colones aides pour la lisibilité du code
%	
	\subsection{Équation simple}
	Définissons en premier lieu le concept d'équation \textit{en-ligne} comme celle-ci $a^2+b^2=c^2$ (théorème de Pythagore) et le concept d'équation  \textit{hors-texte}  tel qu'imprimée ci-dessous 
	\[
	\forall\varepsilon_{>0},\exists\delta : \forall x \in D_f \cap V(x_0,\delta), f(x)\in V\big(f(x_0),\varepsilon\big)
	\]
	\par On préconise l'usage hors-texte pour les équations ou pour les formules les plus importantes et/ou à plusieurs niveaux tels que
%
	\begin{table}
		\centering
		%% Somme supérieur de Darboux
		\begin{equation}
		R^+(f,P):=\sup_{\substack{\xi_j\in I_j \\
				0<j<n }}
		R(f,P,\xi)\label{darboux}
		\end{equation}	
		\hrule
		\begin{verbatim}
		\begin{equation}
		R^+(f,P):=
		\sup_{\substack{\xi_j\in I_j\\ 0<j<n }}
		R(f,P,\xi)
		\end{equation}
		\end{verbatim}
		\hrule
	\end{table}
%
	\par Les possibilités qu'offrent les extensions mathématiques sont presque infinies. Voyez cet exemple comme un aperçue de la flexibilité de \LaTeX{} pour produire d'exceptionnel autant que complexe, équation (aussi loufoque que l'équation (\ref{loufoque})).
%
	\begin{table}
		\setcounter{equation}{3}
		\centering
		\[
		\int_{\int_{a}^{\int_{a}^{b}}}^{c}x\,dx\, 
		\tag{\theequation}\label{loufoque}
		\]
		\hrule
		\begin{verbatim}
			\[
			\int_{\int_{a}^{\int_{a}^{b}}}^{c}x\,dx
			\tag{\theequation}\label{loufoque}
			\]
		\end{verbatim}
		\hrule
	\end{table}
%
	\par Cette formule se trouve dans l'environnement \textit{equation} sous sa forme abrégée, c.-à-d. qu'elle est contenue dans \verb|\[...\]| tandis que le théorème de Pythagore est simplement en mode mathématique, c.-à-d. qu'elle est inscrite entre  \verb|$...$|. À noter que pour (\ref{loufoque}) l'environnement \textit{equation} serait plus adéquat du fait que l'expression est référencée (les références croisées sont traitées à la page \ref{crossreferenncing}). D'ailleurs, il parfois plus sage d'utiliser l'environnement au lieu de sa version compacte pour des raisons de lisibilité du code. En revanche, si l'équation n'a pas à être référencée et qu'elle est plus ou moins complexe, alors il faut utiliser \textit{equation*}. La particularité du mode mathématique est que toutes les lettres sont prises comme des variables et sont ainsi imprimées en italique, mais souvent les fonctions sont en police droite. Ce sont les extensions de \textit{AMS} qui fournissent la plupart des fonctions populaires dont la police est droite ($\sin \neq sin$). Voir la Table~\ref{amsfunction} résumant les principales fonctions de l'\AmS. Si une fonction est absente des extensions mentionnées plus haut, il est possible de créer des fonctions de même qu'il est possible de créer des opérateurs n-aires en ajoutant dans le préambule des commandes semblables à celle-ci (voir l'équation~\ref{newoperator}). Il est aussi possible définir des opérateurs simples via \verb|newcommand| seulement dans le cas opérateurs n-aires les bornes seront plus difficile à manipuler d'autant qu'il faut le définir en police droite.  
%
		\[\verb|\DeclareMathOperator{command}{definition}|\]
%
	\par Voici un exemple d'utilisation tiré du \textit{lshort}\footnote{\url{http://mirrors.ibiblio.org/CTAN/info/lshort/french/lshort-fr.pdf}} de ces commandes dont la particularité de la déclaration étoilée est le fait de créer un opérateur n-aire.
%	
	\begin{table}
		%Exemple du lshort
		\begin{equation}\label{newoperator}
		3\argh = 2\Nut_{x=1}
		\end{equation}
		\hrule
		\begin{verbatim}
			%\DeclareMathOperator{\argh}{argh}
			%\DeclareMathOperator*{\Nut}{Nut}
			\begin{equation*}
			3\argh = 2\Nut_{x=1}
			\end{equation*}
		\end{verbatim}
		\hrule
	\end{table}
%
	\subsection{Comment faire des équations mathématiques}
	La présentation de l'environnement \textit{equation} terminé peut passer à l'élaboration de formules et d'expressions plus complexes à l'image l'équation (\ref{darboux}). Il y a d'abord les indices et les exposants qui sont respectivement positionnés par \_ et \^~($a^2=$\verb|a^2|). S'il y a plus d'un caractère à être mis en exposant ou en indice, ceux-ci sont alors regroupés entre accolades \verb|{...}|.
	\[e^\pi i\neq e^{\pi i}\qquad\verb|e^\pi i\neq e^{\pi i}|\]
	\par Ce groupement entre accolades est fait dès qu'il y a plus d'un élément à mettre dans un opérateur tel que la racine n\ieme{} (ou toute autre opération ainsi que les fonctions cf.~Table~\ref{amsfunction}):
	\[\sqrt abc\neq \sqrt{abc}\qquad\verb|\sqrt abc\neq \sqrt{abc}|\]
	\par La commande \verb|\surd| permet d'obtenir seulement le symbole de la racine carrée. Similairement aux variables, les opérateurs \textit{n-aires} (à différentier des opérateurs binaires Eg.~\verb|+|) telles que la sommation utilise la même syntaxe pour les bornes (voir l'équation~\ref{lfourier}).
%
	\begin{table}
		\centering
		\begin{equation}\label{lfourier}
			a_n=\frac{1}{L}\int_{-L}^{L}f(x)\cos
			\frac{n\pi x}{L}\,dx,\,n=1,2,\dots
		\end{equation}
		\hrule
		\begin{verbatim}
			\begin{equation}
			a_n=\frac{1}{L}\int_{-L}^{L}f(x)\cos
			\frac{n\pi x}{L}\,dx,\,n=1,2,3,\dots
			\end{equation}
		\end{verbatim}
		\hrule
	\end{table}
%
	\subsubsection{Formules multiples}
	Souvent lors de manipulations arithmétiques, il est requis d'écrire une série d'égalité (ou inégalité) jusqu'à la réponse simplifiée. Il existe plusieurs environnements permettant d'écrire des formules sur plus d'une ligne notamment \verb|align, eqnarray, array, gather| ainsi que leur commande étoilée, mais ces dernières sont rapidement difficiles à utiliser lorsque les membres de l'équation sont trop longs.
%
	\begin{table}
		\centering
		%% Exemple du lshort
		\begin{eqnarray}
			a & = & b + c     \\
			& = & d + e + f + g + h^2 + i^2
			+j + k + l + o
			\label{eq:eqnarrayfautif}
		\end{eqnarray}
		\hrule
		\begin{verbatim}
			\begin{eqnarray}
			a & = & b + c     \\
			& = & d + e + f + g + h^2+i^2
			+j + k + l + o
			\end{eqnarray}		
		\end{verbatim}
		\hrule
	\end{table}
%
	\par Pour essayer de régler le problème, simplifier les équations est souvent la solution la plus facile, mais dans le cas où c'est impossible, l'environnement \verb|IEEEeqnarray| \footnote{\url{https://ras.papercept.net/conferences/support/files/IEEEtran_HOWTO.pdf}} propose une panoplie d'outils et d'options pour les équations multilignes (à ne pas confondre avec l'environnement \verb|multline| qui est utilisé lorsque les équations sont démesurément longues pour tenir sur une ligne). Les prochains exemples font un bref survole de \verb|IEEEeqnarray|, mais ne sont nullement représentatif de toutes les possibilités qu'offre l'extension. Pour avoir accès à cedit environnement, il suffit d'entrer dans le préambule la commande suivante avec l'option \verb|\usepackage[retainorgcmds]{IEEEtrantools}|. Malgré tout, il est possible d'avoir des chevauchements similaires l'équation~\ref{eq:eqnarrayfautif}. Le cas échéant, l'on peut couper via \verb|\\| et aligner avec \verb|&| de la même façon que pour environnement \verb|eqnarray|. Les membres de gauches et de droite respectivement, sont plus proche du symbole central, celui entre \verb|&...&|.
%
	\begin{table}
		\centering
		%% Exemple modifié du lshort
		\begin{IEEEeqnarray}{rCl}
			a & = & b + c       \\
			& = & d + e + f + g + h^2+i^2
			+j + k + l + n + o
			\IEEEeqnarraynumspace
			\label{eq:IEEEeqnarrayfautif}
		\end{IEEEeqnarray}
		\hrule
		\begin{verbatim}
			\begin{IEEEeqnarray}{rCl}
			a & = & b + c       \\
			& = & d + e + f + g + h^2+i^2
			+j + k + l + n + o
			\IEEEeqnarraynumspace
			\end{IEEEeqnarray}
		\end{verbatim}
		\hrule
	\end{table}
%
	\par Cependant, le résultat n'est toujours pas optimal puisque la longueur du membre de gauche \ref{eq:IEEEeqnarrayfautif} n'est pas balancée par rapport à l'équation précédente. Les numéros d'équations ont été poussé à l'extérieur de la colonne ce qui est aussi un problème. Diviser le membre en plusieurs parties permet de corriger cette coquille esthétique.
%
	\begin{table}
		\centering
		%% Exemple modifié du lshort
		\begin{IEEEeqnarray}{rCl}
			a & = & b + c\\
			& = & d + e + f + g + h^2+i^2\nonumber\\
			&   & +\:j + k + l + n + o
			\label{eq:IEEEeqnarraycorrect}
		\end{IEEEeqnarray}
		\hrule
		\begin{verbatim}
			\begin{IEEEeqnarray}{rCl}
			a & = & b + c\\
			&=&d + e + f + g + h^2+i^2\nonumber\\
			& & +\:j + k + l + n + o
			\end{IEEEeqnarray}
		\end{verbatim}
		\hrule
	\end{table}
%
	\par Puisque l'addition est un opérateur binaire,non unaire ($\displaystyle{2-2}$ vs $\displaystyle{-2}$) d'où l'importance d'ajouter un espacement (avant la variable) via la commande \verb|\:| lorsque la partie de la formule commence par un opérateur.	 
%
	%%%%%%%%%%%%%%%%%%%%%%%%%%%%%%%%%%%%%%%%%%%%%%%%%%%%%%%%%%%%%%%%%%%%%%%%%%%%
	%%%%%%%%%%%%%%%%%%%%%%%%%%%%%%%%%%%%%%%%%%%%%%%%%%%%%%%%%%%%%%%%%%%%%%%%%%%%
	\begin{table*}
		\begin{tabular}{clclcl}
			\hline
			      $\alpha$        & \verb|\alpha|                     &        $\iota$        & \verb|\iota|                      & $\sigma\varsigma$ & \verb|\sigma|\verb|\varsigma| \\
			       $\beta$        & \verb|\beta|                      &   $\kappa\varkappa$   & \verb|\kappa|\verb|\varkappa|     &      $\tau$       & \verb|\tau|                   \\
			      $\gamma$        & \verb|\gamma|                     &       $\lambda$       & \verb|\lambda|                    &    $\upsilon$     & \verb|\upsilon|               \\
			      $\delta$        & \verb|\delta|                     &         $\mu$         & \verb|\mu|                        &   $\phi\varphi$   & \verb|\phi\varphi|            \\
			$\epsilon\varepsilon$ & \verb|\epsilon|\verb|\varepsilon| &         $\xi$         & \verb|\xi|                        &      $\chi$       & \verb|\chi|                   \\
			       $\zeta$        & \verb|\zeta|                      &         $\nu$         & \verb|\nu|                        &      $\psi$       & \verb|\psi|                   \\
			       $\eta$         & \verb|\eta|                       &      $\pi\varpi$      & \verb|\pi|\verb|\varpi|           &     $\omega$      & \verb|\omega|                 \\
			  $\theta\vartheta$   & \verb|\theta| \verb|\vartheta|    &     $\rho\varrho$     & \verb|\rho|\verb|\varrho|         &                   &                               \\
			                      &                                   &                       &                                   &                   &                               \\
			  $\Gamma\varGamma$   & \verb|\Gamma|\verb|\varGamma|     &      $\Xi\varXi$      & \verb|\Xi|\verb|\varXi|           &   $\Phi\varPhi$   & \verb|\Phi|\verb|\varPhi|     \\
			  $\Delta\varDelta$   & \verb|\Delta|\verb|\varDelta|     &      $\Pi\varPi$      & \verb|\Pi|\verb|\varPi|           &   $\Psi\varPsi$   & \verb|\Psi\varPsi|            \\
			  $\Theta\varTheta$   & \verb|\Theta|\verb|\varTheta|     &   $\Sigma\varSigma$   & \verb|\Sigma|\verb|\varSigma|     & $\Omega\varOmega$ & \verb|\Omega\varOmega|        \\
			 $\Lambda\varLambda$  & \verb|\Lambda|\verb|\varLambda|   & $\Upsilon\varUpsilon$ & \verb|\Upsilon|\verb|\varUpsilon| &                   &                               \\ \hline
		\end{tabular}
		\caption{Alphabet grec}
		\label{greekLetters}
	\end{table*}
	% Pour alléger le code on définit de nouveaux types de colonne
	% Voici le code dans le préambule
	%\newcolumntype{L}{>{$}l<{$}}
	%\newcolumntype{V}{>{\ttfamily\textbackslash}l}
	\begin{table*}
		\centering
		\begin{tabular}{LVLVLVLV}
			\hline
			\arccos & arccos & \arctan & arctan & \arcsin & arcsin & \arg    & arg    \\
			\cos    & cos    & \cot    & cot    & \cosh   & cosh   & \coth   & coth   \\
			\csc    & csc    & \det    & det    & \deg    & deg    & \dim    & dim    \\
			\exp    & exp    & \hom    & hom    & \gcd    & gcd    & \inf    & inf    \\
			\ker    & ker    & \lim    & lim    & \lg     & lg     & \liminf & liminf \\
			\limsup & limsup & \log    & log    & \ln     & ln     & \max    & max    \\
			\min    & min    & \sec    & sec    & \Pr     & Pr     & \sin    & sin    \\
			\sinh   & sinh   & \tan    & tan    & \sup    & sup    & \tanh   & tanh   \\
			\hline
		\end{tabular}
		\caption{Principales fonctions de \AmS-\LaTeX}
		\label{amsfunction}
	\end{table*}
%
	\begin{table*}
		\centering
		\begin{tabular}{LVLVLV}
			\hline
			\int      & int      & \bigcap   & bigcap   & \prod      & prod      \\
			\iint     & iint     & \bigcup   & bigcup   & \coprod    & coprod    \\
			\iiint    & iiint    & \biguplus & biguplus & \bigodot   & bigodot   \\
			\iiiint   & iiiint   & \bigsqcup & bigsqcup & \bigoplus  & bigoplus  \\
			\oint     & oint     & \bigwedge & bigwedge & \bigotimes & bigotimes \\
			\idotsint & idotsint & \bigvee   & bigvee   & \sum       & sum       \\ \hline
		\end{tabular}
		\caption{Opérateur n-aires}
		\label{operator}
	\end{table*}
	%%%%%%%%%%%%%%%%%%%%%%%%%%%%%%%%%%%%%%%%%%%%%%%%%%%%%%%%%%%%%%%%%%%%%%%%%%%%
	%%%%%%%%%%%%%%%%%%%%%%%%%%%%%%%%%%%%%%%%%%%%%%%%%%%%%%%%%%%%%%%%%%%%%%%%%%%%
%
	\subsubsection{Fonctions par parties, matrices et tableaux}
	%			\substack
	%			délimiteur (invisible)
	%			Manipuler les polices mathématiques
	%			Théorèmes, lemmes, etc.  
	\subsection{Théorèmes, définitions, preuves}
%
	%			En-têtes améliorés
	\section{Objets flottants}
	D'une manière générale plus il y a d'images, de tableaux et de graphiques plus il est difficile de gérer l'alignement de ces objets tout en gardant les références croisées à jour. Ce constat est surtout vrai sur des logiciels de traitement de texte tels que \textit{LibreOffice} et la \textit{Suite Office}. Cela est, presque, facile sous \LaTeX, puisqu'il gère ces différents éléments de manière à rendre le document harmonieux. La figure~\ref{fig:pinghistogramtue-5-may-2020-084447} est directement entrée. 
	\begin{figure}
		\centering
		\includegraphics[width=0.7\linewidth]{"PingHistogramTue 5 May 2020 084447"}
		\caption[PingTimes]{Histogramme du temps aller-retour des requêtes envoyé aux serveurs de Google.ca à travers un réseau IP.}
		\label{fig:pinghistogramtue-5-may-2020-084447}
		\hrule
		\begin{verbatim}
			\begin{figure}[tph!]
			\centering
			\includegraphics[width=0.7\linewidth]%
			{"PingTime"}
			\caption[PingTimes]{Histogramme %
			du temps aller-retour des %
			requêtes envoyés aux serveur%
			de Google.ca à travers un réseau IP.}
			\end{figure}
		\end{verbatim}
		\hrule
	\end{figure}
%	
	\onecolumn
%
	\par Avec les graphiques, il parfois nécessaire de changer orientation de la page de même que sa structure, passer de deux colonnes à une seule. Le changement d'orientation est fourni par l'extension \verb|lscape| (à inclure dans le préambule). Les commandes \verb|\onecolumn| et \verb|\twocolumn| permette de changer la structure en de la page. L'extension \verb|mulitcols|, il est possible d'avoir neuf colonnes sur une page (cette extension fournit l'environnement\footnote{\url{http://ctan.mirror.colo-serv.net/macros/latex/required/tools/multicol.pdf}} \verb|multicols|).
	
	
		
%
	\begin{figure}[H]
		\centering
		\subfloat[Histogramme du temps aller-retour des requêtes envoyé aux serveurs de Google.ca à travers un réseau IP]{%
			{\includegraphics[width=0.4\linewidth]{"PingHistogramTue 5 May 2020 084447"}}%
		}%
		\qquad
		\subfloat[Le graphique des quantiles normaux de la distribution]{%
			{\includegraphics[width=0.4\linewidth]{"PingQuantilePlotTue 5 May 2020 084447"}}%
		}%
		\qquad
		\subfloat[Évolution du pingtime dans le temps (ordre chronologique)]{%
			{\includegraphics[width=0.4\linewidth]{"PingListPlotTue 5 May 2020 084448"}}%
		}%
		\caption{Résultat d'un algorithme sur Mathematica}%
		\label{fig:example}%
		\hrule
		\begin{verbatim}
		\begin{figure}[!htp]%
		\centering
		\subfloat[Histogramme du temps aller-retour des requêtes
		envoyés aux serveurs de Google.ca à travers un réseau IP]{%
		{\includegraphics[width=0.4\linewidth]{image1}} }\qquad
		\subfloat[Le graphique des quantiles normaux de la distributions]{%
		{\includegraphics[width=0.4\linewidth]{image2}} }\qquad
		\subfloat[Évolution du pingtime dans le temps (ordre chronologique)]{%
		{\includegraphics[width=0.6\linewidth]{image3}} }%
		\caption{Résultat de l'algorithme sur Mathematica}%
		\end{figure}
		\end{verbatim}
		\hrule
	\end{figure}
	%https://www.latex-project.org/help/documentation/amsldoc.pdf
	%http://mirrors.ibiblio.org/CTAN/info/lshort/french/lshort-fr.pdf
	%https://cs.overleaf.com/learn
\end{document}
