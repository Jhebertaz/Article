\section{Objets flottants et références croisées}
L'organisation d'un document ne se résume pas seulement à sa structure textuelle, les différents complément telle la pagination, la table des matière et même l'index. L'index, par exemple, repose sur une classification alphabetique des \textit{termes}\footnotemark\footnotetext{mots, concepts, objets,...} clés d'un ouvrage. L'allusion à un élément apparaissant dans le document est appelé référence croisée. Elle sont largement utilisé sous \LaTeX{} surtout en présence d'objets flottants. Les flottant, communément appelé \textit{floats}, sont des objects statiques ne pouvant être \og brisé\fg{}. D'une manière générale plus il y a d'images, de tableaux et de graphiques plus il est difficile de gérer l'alignement de ces objets tout en gardant les références croisées à jour.
% Les références croisées servent à indiquer  lecteur un objet flottant, une formule et, etc. en étiquettant l'objet.
Ce constat est surtout vrai sur des logiciels de traitement de texte tels que \textit{LibreOffice} et la \textit{Suite Office}. Cela est, presque, facile sous \LaTeX, puisqu'il gère ces différents éléments de manière à rendre le document harmonieux (sans prise de tête pour l'automatisation de la numérotation). Pour commencer, l'on doit mettre certaine extension dans le préambule \textit{graphicx, subfig, float} et \textit{caption} qui respectivement servent à l'insertion d'image, à la manipulation (références) des \og petits \fg{} flottant dans une unique figure, à améliorer l'interface pour la défintion des objets flottants et à personnaliser les légendes dans les environnements flottants. Le fonctionnement de l'environnement flottant \textit{figure} va comme suit:
\begin{figure}[tph!]
	\begin{verbatim}
	\begin{figure}[tph!]
	    \centering
	    \includegraphics[width=1\linewidth]%
	        {image}
	    \caption[Cour texte]{Long texte}
	    \label{fig:image}
	\end{figure}
	\end{verbatim}
\end{figure}\linebreak
%
Les lettres qui sont placées entre crochets sont les paramètres de position de la figure dont une liste détaillée se trouve sur Wikibook\footnotemark \footnotetext{\url{https://fr.wikibooks.org/wiki/LaTeX/Éléments_flottants_et_figures}}. Concernant les références croisées le fonctionnement est encore plus simple : l'on ajoute une étiquette avec la commande \verb|\label{|\textit{marqueur}\verb|}| pour ensuite la référencer avec \verb|\ref{|\textit{marqueur}\verb|}|. Les références croisées sont importantes en raison de la manière dont \LaTeX{} gère le positionnement des objet flottant. Ces derniers peuvent être situé ailleurs qu'a leur position relative au code. 
\par 
%
%\onecolumn
%
%\par Avec les graphiques, il parfois nécessaire de changer orientation de la page de même que sa structure, passer de deux colonnes à une seule. Le changement d'orientation est fourni par l'extension \verb|lscape| (à inclure dans le préambule). Les commandes \verb|\onecolumn| et \verb|\twocolumn| permette de changer la structure en de la page. L'extension \verb|mulitcols|, il est possible d'avoir neuf colonnes sur une page (cette extension fournit l'environnement\footnote{\url{http://ctan.mirror.colo-serv.net/macros/latex/required/tools/multicol.pdf}} \verb|multicols|).
%
%%
%\begin{figure}[H]
%	\centering
%	\subfloat[Histogramme du temps aller-retour des requêtes envoyé aux serveurs de Google.ca à travers un réseau IP]{%
%		{\includegraphics[width=0.4\linewidth]{"PingHistogramTue 5 May 2020 084447"}}%
%	}%
%	\qquad
%	\subfloat[Le graphique des quantiles normaux de la distribution]{%
%		{\includegraphics[width=0.4\linewidth]{"PingQuantilePlotTue 5 May 2020 084447"}}%
%	}%
%	\qquad
%	\subfloat[Évolution du pingtime dans le temps (ordre chronologique)]{%
%		{\includegraphics[width=0.4\linewidth]{"PingListPlotTue 5 May 2020 084448"}}%
%	}%
%	\caption{Résultat d'un algorithme sur Mathematica}%
%	\label{fig:example}%
%	\hrule
%	\begin{verbatim}
%	\begin{figure}[!htp]%
%	\centering
%	\subfloat[Histogramme du temps aller-retour des requêtes
%	envoyés aux serveurs de Google.ca à travers un réseau IP]{%
%	{\includegraphics[width=0.4\linewidth]{image1}} }\qquad
%	\subfloat[Le graphique des quantiles normaux de la distributions]{%
%	{\includegraphics[width=0.4\linewidth]{image2}} }\qquad
%	\subfloat[Évolution du pingtime dans le temps (ordre chronologique)]{%
%	{\includegraphics[width=0.6\linewidth]{image3}} }%
%	\caption{Résultat de l'algorithme sur Mathematica}%
%	\end{figure}
%	\end{verbatim}
%	\hrule
%\end{figure}
%https://www.latex-project.org/help/documentation/amsldoc.pdf
%http://mirrors.ibiblio.org/CTAN/info/lshort/french/lshort-fr.pdf
%https://cs.overleaf.com/learn