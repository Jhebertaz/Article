\section*{Article}
Les motivations premières d'un article sont la recherche et le partage d'une question ainsi que de sa réponse. De même un corrigé à bien plus de valeur si les solutions sont données avec les énoncées. Cela est bien, mais cette rigueur n'a pas été utilisée par un certain Srinivasa Ramanujan assurément l'autodidacte le plus productif des 20 siècles. Ce mathématicien de génie aura tout au long de sa carrière découvert pas moins de 3900 formules et théorèmes, mais la quasi-totalité est léguée sans démonstration. L'on nomme d'ailleurs les Cahiers de Ramanujan, les quatre recueils manuscrits où se retrouvent ses résultats. L'histoire ne s'arrête pas à sa mort, car plus de 50 ans après celle-ci, un autre mathématicien, Bruce Carl Berndt, s'intéresse de près à ces cahiers et consacra plus de 20 ans à leurs éditions commentées. Ce dernier ainsi que ses collaborateurs George Andrews, Richard Askey et Robert A. Rankinse  tenteront de prouver ou de retracer les références expliquant les résultats de Ramanujan. Maintenant, la question ayant poussé l'écriture du présent article est la suivante : est-il possible de prouver une des formules (aux allures spectaculaires) en utilisant des outils simples? La formule est la suivante (présente sur la page Wikipédia consacrée à Ramanujan):
\begin{corollary}
	\begin{equation*}
		{\displaystyle {\sqrt {5+{\sqrt {5+{\sqrt {5-{\sqrt {5+{\sqrt {5+{\sqrt {5+\cdots }}}}}}}}}}}}={\frac {2+{\sqrt {5}}+{\sqrt {15-6{\sqrt {5}}}}}{2}}}
	\end{equation*}
	où l'alternance des signes est périodique (\(+,+,+,-,+\)).
\end{corollary}
Cette identité est en fait un corolaire de la proposition suivante:
\begin{proposition}[\og Entry 32\fg{}]
	Les inégalités suivantes:
	\begin{align*}
		x^2&=a+y\\
		y^2&=a+z\\
		z^2&=a+u\\
		u^2&=a+x
	\end{align*}
	détermine un polynôme en \(x\) (ou \(y\) ou \(z\) ou \(u\)) de degré 16. Ce polynôme peut-être factorisé en un produit de quatre polynômes quartique, l'un d'entre eux est \(x^4-2ax^2-x +a^2-a\). Les autres polynômes quartiques restantes ont la forme 
	%	\begin{equation}
	%		\left(x^2+px+\frac{1}{2}\left\{p^2-2a-\frac{1}{p}\right\}\right)\left(x^2+qx+\frac{1}{2}\left\{q^2-2a-\frac{1}{q}\right\}\right),
	%	\end{equation}
	\begin{equation}\label{eq:1}
		(x^2+px+\frac{1}{2}(p^2-2a-{1/p}))(x^2+qx+\frac{1}{2}(q^2-2a-{1/q})),
	\end{equation}
	où \(pq=-1\) et \(p+q\) sont une racine de l'équation polynomiale
	\begin{equation}\label{eq:2}
		z^3+3z=a(1+az)
	\end{equation}
\end{proposition}
%\begin{remark}
%	La partie fractionnaire de \(x\) est notée \(\{x\}\in [0,1)\) tandis que la partie entière de \(x\) est notée \([x]\in\mathbb{Z}\)
%\end{remark}
\begin{proof}[Démonstration : Direct]
	La preuve qui suivra est tirée du quatrième volume des ouvrages de Berndt.\\
	Soit \(\{a_n\}\) une suite de terme telle que \(a_1=\sqrt{5}, a_2=\sqrt{5+\sqrt{5}},...\) où \(a_n\) est la \(n\)\ieme{} radical imbriqué de la partie de droite du corollaire. Maintenant posons les fonctions \(f,g\) continues définies par
	\[
	f(x)=\sqrt{5+\sqrt{5+\sqrt{5-x}}}\quad\text{et}\quad g(x)=\sqrt{5+x}\pod{0\leq x\leq 5}
	\]
	Remarquons que \(f\) est décroissante et que \(g\) est croissante sur leur interval de définition. En particulier,
	\begin{align*}
		f(0)=a_3>f\left(\sqrt{5}\right)&=(f\circ g)(0) = a_4>(f\circ g)\left(\sqrt{5}\right)\\
		=a_5>&(f\circ g)\left(\sqrt{5+\sqrt{5}}\right)=a_6>(f\circ g)(a_3)=a_7
	\end{align*}
	et 
	\begin{align*}
		a_7<(f\circ g)(a_4)=a_8<(f\circ g)(a_5)&=a_9<(f\circ g)(a_6)\\
		=a_{10}<&(f\circ g)(a_7)=a_{11}
	\end{align*}
	Il suit par induction que 
	\begin{equation}\label{eq:inequalities}
		\begin{cases}{}
			a_{8n+3}>a_{8n+4}>a_{8n+5}>a_{8n+6}>a_{8n+7}\\
			a_{8n+7}<a_{8n+8}<a_{8n+9}<a_{8n+10}<a_{8n+11}
		\end{cases}
	\end{equation}
	pour tout entier \(n\) non négatif. En prenant la racine carrée et puis en additionnant ou soustrayant 5, l'on montre aussi par induction que
	\(2<a_7<a_{15}<a_{23}<\cdots<a_{8n+7}<a_{8n+11}<a_{8n+3}<\cdots<a_{19}<a_{11}<a_3<5\), pour tout entier \(n\) non négatif. Par conséquent, la sous-suite \(\{a_{8n+7}\}\) est monotone croissante tandis que la sous-suite \(\{a_{8n+3}\}\) est monotone décroissante. Toutes deux sont bornées, alors elle converge, et disons qu'elles convergent respectivement vers \(\alpha\) et \(\beta\). Notons que \(a_{8n+7}=(f\circ g)(a_{8n+3})\). Par continuité de \(f\) et \(g\)
	\begin{align*}
		\lim_{n\to\infty}(f\circ g)(a_{8n+3}) & =(f\circ g)\left(\lim_{n\to\infty}a_{8n+3}\right) \\
		& =(f\circ g)(\beta),
	\end{align*}
	alors que \(a_{8n+7}\) tend vers \(\alpha\) lorsque \(n\to\infty\). Aussi puisque \(a_{8n+11}=(f\circ g)(a_{8n+7})\), alors par un argument similaire, \(\beta=(f\circ g)(\alpha)\).\\
	Donc, \(\beta=(f\circ g)(\beta)\). En posant la fonction \(h=f\circ g\), alors on observe que \(\alpha\) et \(\beta\) sont des points fixes de \(h\circ h\), c'est-à-dire que \(p(x)=x\) où dans ce cas \((h\circ h)(\alpha)=\alpha\) et \((h\circ h)(\beta)=\beta\). En dérivant \((h\circ h)\), on remarque \(0<(h\circ h)'(x)<1\) pour \(a_7\leq x\leq a_3\). Ainsi, l'équation \((h\circ h)(x)=x\) a au plus une racine sur cet intervalle. Alors, \(\alpha=\beta\) et donc la sous-suite \(\{a_{4n+3}\}\) convergent.\\
	Par \eqref{eq:inequalities}, on peut aussi conclure que la suite \(\{a_n\}\) converge.\\
	en posant, \(a=5\) dans la suite d'équalité de la proposition présentée plus haute, on obtient
	\begin{align*}
		u^2&=5+x,\\
		z^2&=5+u,\\
		y^2&=5+z,\\
		x^2&=5+y,		
	\end{align*}
	Que l'on peut combiner de sorte à obtenir l'équation polynomiale
	\begin{equation}\label{eq:4}
		5+x=(((x^2-5)^2-5)^2-5)^2
	\end{equation}
	En posant, \(F(x)=5+x-(((x^2-5)^2-5)^2-5)^2\) et en le factorisant par \textit{Mathematica} on trouve
	\begin{multline*}
		\left(x^2-x-5\right)\cdot\left(x^2+x-4\right)\cdot\left(x^4-4 x^3-4 x^2+31 x-29\right)\cdot\\
		\left(x^8+4 x^7-10 x^6-54 x^5+9 x^4+226 x^3+125 x^2-301 x-269\right)
	\end{multline*}
	Un monstre dont l'une des racines est la limite de \(\{a_n\}\). Par la proposition, \(p+q\) est une racine de \eqref{eq:2}
	\begin{equation*}
		z^3-17z-4=(z^2-4z-1)(z+4)=0.
	\end{equation*}
	Soit \(p+q=-4\). Alors puisque \(pq=-1\),on trouve que \(p,q=-2\pm\sqrt{5}\). Ainsi, le polynôme de l'équation \eqref{eq:1} est donné par
	\begin{align*}
		\left(x^2+\left(-\sqrt{5}-2\right) x+\frac{1}{2} \left(\left(-\sqrt{5}-2\right)^2-10-\frac{1}{-\sqrt{5}-2}\right)\right)\cdot\\
		\left(x^2+\left(\sqrt{5}-2\right) x+\frac{1}{2} \left(\left(\sqrt{5}-2\right)^2-10-\frac{1}{\sqrt{5}-2}\right)\right)=\\
		x^4-4 x^3-4 x^2+31 x-29
	\end{align*}
	Les quatre racines de ce polynôme sont données par 
	\(\frac{2\pm\sqrt{5}\pm\sqrt{15-6\sqrt{5}}}{2}\). Il ne reste plus qu'à montrer que \(\frac{2+{\sqrt {5}}+{\sqrt {15-6{\sqrt {5}}}}}{2}\) à l'expansion radicale donnée dans l'énoncé du corollaire. Cela peut se faire à l'aide de \textit{Mathematica} en calculant les seize racines de \(F(x)\) dont une seule satisfait l'équation requise (une variante de \ref{eq:4}):
	\[(f\circ g)(x)=x=\sqrt{5+\sqrt{5+\sqrt{5-\sqrt{5+x}}}}\]
\end{proof}
En outre la preuve a utilisé la convergence de suites ainsi que de sous-suites, la continuité, la différentialiste, la définition de point fixe et le principe d'induction. L'utilisation de ces concepts est restée relativement simple, tout en montrant une identité biscornue. Évidemment, la preuve ne serait pas aussi directe sans l'entrée \#32 dont la démonstration n'est pas plus difficile à comprendre malgré que travailler avec des polynômes de degré seize mène inévitablement à jongler avec de grandes équations.